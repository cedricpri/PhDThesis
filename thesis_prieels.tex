\documentclass[a4paper, 11pt]{report}
\usepackage[top=3cm, bottom=3cm, left=3cm, right=3cm]{geometry}
\usepackage{graphicx}
\usepackage{booktabs}
\usepackage{url}
\usepackage[english]{babel}
\usepackage[latin1]{inputenc}
\usepackage{hyperref}
\hypersetup{
    colorlinks,
    citecolor=black,
    filecolor=black,
    linkcolor=black,
    urlcolor=black
}
\usepackage{mathenv}
\usepackage{amsmath}
\usepackage{color}
\usepackage{caption}
\usepackage[bottom]{footmisc}
\usepackage{cancel}
\usepackage{multirow}
\usepackage[toc,page]{appendix}
\usepackage[]{acronym}
\usepackage{setspace}
%\usepackage[Sonny]{fncychap}
\usepackage{fancyhdr}
\usepackage{titlesec}

\makeatletter
\titleformat{\chapter}[block]
  {\Huge}{\filright\enspace \@chapapp~\thechapter\enspace \\}
  {8pt}{\Huge\bfseries\filcenter}
\titlespacing*{\chapter}
  {0pt}{-70pt}{20pt}
\makeatother

%\setlength{\headheight}{20pt}
%\pagestyle{fancy}
%\renewcommand{\chaptermark}[1]{\markboth{#1}{#1}}
%\rhead{\thepage \hfill \leftmark}
%\lhead{\thepage \hfill \leftmark}
%\cfoot{}

\begin{document}

%Titlepage
\doublespacing
\baselineskip=0.8\baselineskip
\setlength{\parindent}{0pt}	

\pagenumbering{roman}
\begin{titlepage}

	\centering
	\includegraphics[width=0.15\textwidth]{figs/image_UC.png}\par
	{\scshape\LARGE Facultad de Ciencias \par}
	
	\vspace{1.5cm}
	
	%English title
	\noindent\rule{15cm}{0.4pt}\par 
	{\huge\bfseries Search for dark matter production in association with top quarks in the dilepton final state at $\sqrt{s} = $ 13 TeV\par}
	\noindent\rule{15cm}{0.4pt}\par 
	
	{\Large A thesis submitted in fulfilment of the requirements for the \par \LARGE \textbf{Degree of Doctor of Philosophy} \par \noindent\rule{15cm}{0.4pt}}
	
	\vspace{1.2cm}
	{\Large Written by \\ \textbf{C�dric Prie�ls}\par}
	\vspace{1.0cm}
	{\Large Under the supervision of \\ \textbf{J�natan Piedra G�mez} \\
	\textbf{Pablo Mart�nez Ruiz del �rbol}\\}
	\vspace{1.8cm}
	{\Large June 2020}
	\vfill

\end{titlepage}

%Empty page

%\clearpage
%\thispagestyle{empty}
%\phantom{a}
%\vfill
%\newpage

%Abstract and keywords

\setcounter{page}{1}

\chapter*{\huge{Abstract}}
\newpage
%
\chapter*{\huge{Resumen}}
\newpage
%
\chapter*{\huge{Acknowledgements}}
\newpage
%
\chapter*{\huge{Acronyms used}}
\begin{acronym}

%List all the acronyms used
\acro{SM}[SM]{Standard Model}
\acro{DM}[DM]{Dark Matter}
\acro{LHC}[LHC]{Large Hadron Collider}
\acro{CMS}[CMS]{Compact Muon Solenoid}
\acro{ATLAS}[ATLAS]{A Toroidal LHC ApparatuS}
\acro{CERN}[CERN]{European Council for Nuclear Research}
\acro{CMB}[CMB]{Cosmic Microwave Background}
\acro{ML}[ML]{Machine Learning}
\acro{MFV}[MFV]{Minimal Flavour Violation}
\acro{WIMP}[WIMP]{Weakly Interactive Massive Particle}
\acro{PF}[PF]{Particle Flow}

\end{acronym}
\newpage

%Table of contents
\tableofcontents

\thispagestyle{empty}
\newpage

%Here it starts!
\setlength{\parskip}{10pt}
\pagenumbering{arabic}

%Introduction

\chapter{Introduction}

The \ac{SM} of particle physics is nowadays the most accepted mathematical model used to describe the elementary particles and three of the four fundamental forces of nature (electromagnetic, weak and strong interactions). This model is quite simple in concept, but has been able to describe most of the phenomena observed in nature so far with an incredible level of precision, and made a lot of predictions that have now been proven to be true, such as the postulate of the Higgs mechanism \cite{HiggsPostulate1, HiggsPostulate2} followed by the discovery of the Higgs boson itself in 2012 \cite{HiggsDiscovery1, HiggsDiscovery2} by the \ac{CMS} and \ac{ATLAS} experiments of the \ac{CERN}. 

However, as accurate as it seems to be, this theory is known to have several shortcomings which require further investigation. Eventual exotic particles which do not fit in the current model could be the sign of new physics and have therefore been extensively searched for over the course of the last decades in order to enhance our understanding of the Universe and all its constituants.

In this context, the first serious \ac{DM} hypothesis was introduced in the 1970s because of gravitational anomalies observed by several astrophysicists, as a way to explain the apparent nonluminous missing mass in the Universe \cite{FirstEvidence}. Indeed, the visible mass in most galaxies appears to be way too low to explain several astrophysical processes, such as the rotation curves of the galaxies \cite{RotationCurves}, which seems to be incompatible with the well established laws of gravitation. Some additional measurements of the gravitational lensing (in the Bullet Cluster, for example \cite{BulletCluster}) and the anisotropies observed in the \ac{CMB} \cite{CMBAnisotropies} can be quoted amongst other evidences for the existence of \ac{DM}, as explained in details in Section~\ref{section:DMOrigins}.

As far as we currently know from cosmological measurements, ordinary baryonic matter only constitutes around 5\% of the Universe, while  \ac{DM} represents around 26\% of the energy density of the Universe (the rest is being considered as dark energy) \cite{Repartition}. Understanding the nature and properties of this new kind of exotic matter is therefore crucial to try and understand the physics in the Universe. 

Nowadays, the existence of \ac{DM} is well established in the physics community, even though it has never been observed directly, since our only evidences so far for its existence come from its large-scale gravitational effecs. While its nature is still unknown and extensively studied, one of the best \ac{DM} candidate is the so-called \ac{WIMP}s, predicted to interact both gravitationnaly and weakly with \ac{SM} particles. This would allow direct and indirect direction of such candidates, used as the driving process of many of experiments over the last decades, trying to find the hint of a possible interaction between standard baryonic particles and eventual \ac{DM} particles, or even between several \ac{DM} particles themselves. Dark matter production through the use of a particle accelerator colliding \ac{SM} particles together, such as the \ac{LHC}, is also a possibility, and will be consider as the main channel towards the eventual detection of this exotic matter throughout this work. The production through colliders is actually able to provide constraints on low dark matter masses as well, in a region where both the direct and indirect searches are less efficient, which makes the \ac{LHC} a perfect tool to study this kind of beyond the \ac{SM} particles. 

However, observing \ac{DM} is still extremely difficult, mainly because it barely interacts with ordinary baryonic matter, except through gravity (we have to assume that it does interact with \ac{SM} at least weakly for the sake of this work though, as we would not be able to discover it as an individual particle if it were not the case). This means that nowadays, all the experiments searching for \ac{DM} have only been able to put constraints on the \ac{DM} particle mass and on the interaction cross sections between the dark and standard sectors. Actually, even if the collisions between protons produced by the LHC do have an sufficient amount of energy to produce this kind of particles, we would not expect them to interact with our detector, meaning that a direct detection is out of our hands for now. The eventual presence of such matter has to be inferred from the study of the interaction between \ac{SM} particles and \ac{CMS} itself, since a typical \ac{DM}-like event consists of at least one energetic \ac{SM} particle produced in association with a large imbalance in the transverse momentum due to the presence of an eventual \ac{DM} candidate that was able to escape our detection. %This kind of collider searches can be grouped under the name of \textit{mono-X} searches, where the \textit{X} stands for any kind of \ac{SM} particle (a jet, a lepton or a photon for example). 

In the context of this work, \ac{DM} is searched for in association with one or two top quarks which play the role of the \ac{SM} particle allowing us to trigger the event. This is indeed a perfect channel for this kind of searches if we assume that the interaction between the dark and standard sectors respect the principle of \ac{MFV}, which can be consistently defined independently of the structure of the new physics model \cite{MFVYukawa}. In this case, this interaction should follow the same Higgs-like Yukawa coupling structure as the usual \ac{SM} baryonic particles. This is an important consequence, since it will be shown in Section~\ref{section:DMProperties} that this coupling is actually stronger with more massive particles: the heavier the \ac{SM} particle considered is, the easier it is for this particle to couple with the dark sector. This makes the top quark, the most massive of all the fundamental particles observed by far, is therefore an excellent object to study in this context.

However, this also means that its phenomenology is mostly driven by its large mass and that it decays before hadronization can occur, usually into a W boson and and a bottom quark ($\sim$96\% branching ratio \cite{PDG}). The final state of the process we are intersted in is then be made out of some b jets, leptons and/or quarks and is be categorized depending on this number of b jets and on the decay of the W itself. This work will actually be focused on the two leptons final state, also known as the dileptonic channel, mostly since this channel does not have lots of background processes raising to a similar final state, even though its branching ratio is the smallest, as will be explained in Section~\ref{section:ourChannel}. Additionnaly, leptons are by reconstruction much cleaner than jets. This means that their identification and momentum calculation is easier to perform, and that the uncertainties associated to these measurements will be smaller.

The LHC has now been running for 10 years, and several similar searches have already been carried out and published in the past by the CMS and ATLAS collaborations, at different center of mass energies. First of all, at 8 TeV, several searches for a pair of top quarks were published by the CMS (in association with \ac{DM} in the semileptonic \cite{PreviousDoubleTopSingleLep8CMS} and dileptonic \cite{PreviousDoubleTopDiLep8CMS} final states) and ATLAS collaborations \cite{PreviousDoubleTopAllLep8ATLAS}. At 13 TeV, the ATLAS collaboration published on one hand several studies, considering different final states \cite{PreviousDoubleTopNoLep13ATLAS, PreviousDoubleTopOneLep13ATLAS, PreviousDoubleTopDiLep13ATLAS}. On the other hand, the CMS collaboration published a few extremely important papers for this study \cite{PreviousDoubleTopBottomAllLep13CMS, PreviousDoubleTopAllLep13CMS}. For the first time in 2019, the results obtained by the single top and $t \bar t$ analyses have also been combined and published using the 2016 data \cite{PreviousSingleDoubleTopAllLep13CMS}. Our main objective is now to repeat and improve this analysis that was performed while considering a larger dataset, globally improving the analysis strategy and including the dileptonic final state for the first time in this combination.

After a detailed introduction about \ac{DM} in general, the experimental device will be detailed in Chapter~\ref{chapter:Device}. This will include a discussion about the \ac{LHC} itself, along with a complete description of \ac{CMS}, the detector used to collect the data that will be analyzed throughout this work. This data has been collected during the years 2016, 2017 and 2018 and corresponds to an integrated luminosity of $\sim 138$ fb$^{-1}$, collected during the Run II of operation of the \ac{LHC} and at a center of mass energy $\sqrt{s} = 13$ TeV. In particular, a particular care will be given to the explanation of the \ac{PF} algorithm, used to reconstruct the different objects used and that will be defined in Chapter~\ref{chapter:Reco}, while the estimation of the different backgrounds and the seelction of interesting events will be detailed throughout the Chapters~\ref{chapter:Samples} and ~\ref{chapter:Selection}.

Distinguishing between the signal we are searching for and backgrounds having a much higher cross-section and kinematically really close, such as the \ac{SM} $t \bar t$ without production of \ac{DM} is not a straighforward task (sometimes a production of missing transverse energy due to the presence of physical neutrinos is even obtained). To isolate the signal and to obtain some discriminations between these kind of processes, an algebric recontruction of the event and top-notch \ac{ML} techniques are used in this work, in order to train a network of neurons. The main objective is to make them learn how to combine the discriminating power of a set of input variables in order to create a single ouput variable describing the probability of a single event to be classified as signal or background. All this process will be detailed in the Section~\ref{section:Discrimination} of this work.

Finally, a statistical interpretation of our data will be performed and different sources of systematic uncertainties will be considered in Chapter~\ref{chapter:FinalResults}. This will allow us to set upper limits on the cross section production value of \ac{DM} particles in our particular channel and for the simplified models considered in this analysis. The conclusions of this work and some additional future prospects will then finally be presented in Chapter~\ref{chapter:Conclusion}.
\\

%Empty page

%\clearpage
%\thispagestyle{empty}
%\phantom{a}
%\vfill
%\newpage

\chapter{The Dark Matter case}
\section{At the origins of Dark Matter} \label{section:DMOrigins}
\section{Dark Matter properties} \label{section:DMProperties}
\subsection{Dark Matter candidates}
\subsection{Dark Matter searches} \label{section:DMSearches}
\subsection{Production at the LHC} \label{section:ourChannel}
\subsubsection{The single top production channel} \label{subsection:singleTopChannel}
\subsubsection{The $t \bar t$ production channel} \label{subsection:ttChannel}

\chapter{The experimental device} \label{chapter:Device}
\section{The LHC accelerator}
\section{The CMS detector}
\subsection{Tracker}
\subsection{Electromagnetic calorimeter}
\subsection{Hadronic calorimeter}
\subsection{Muon system}
\subsection{Trigger}
\subsection{Data aquisition}

\chapter{Objects reconstruction} \label{chapter:Reco}
\section{Particle Flow algorithm}
\section{Leptons reconstruction}
\subsection{Electrons}
\subsection{Muons}
\section{Jets reconstruction}
\subsection{B-tagging}
\section{Missing transverse energy}

\chapter{Data, signals and backgrounds} \label{chapter:Samples}
\section{Data samples}
\section{Signal samples}
\section{Background prediction}
\subsection{The main background: $t \bar t$}
\subsubsection{$t \bar t$ reconstruction}
\subsection{Drell-Yan estimation}
\subsection{Non prompt contamination}
\subsection{Smaller bakgrounds}
\subsection{Weights and corrections applied}

\chapter{Event selection} \label{chapter:Selection}
\section{Signal regions}
\section{Control regions}
\section{Background-signal discrimination} \label{section:Discrimination}
\subsection{Dscriminating variables}
\subsection{Neural network}

\chapter{Results and interpretations} \label{chapter:FinalResults}
\section{Systematics and uncertainties}
\section{Results}

\chapter{Conclusions} \label{chapter:Conclusion}
\section{Future prospects}

\begin{appendices}
  
\end{appendices}

\addcontentsline{toc}{chapter}{Bibliography}

\begin{thebibliography}{1}

\bibitem{HiggsPostulate1} 
\href{https://journals.aps.org/prl/abstract/10.1103/PhysRevLett.13.321}{F. Englert and R. Brout, 
"Broken symmetry and the mass of gauge vector mesons",
Phys. Rev. Lett. 13, pp. 321-323, 1964}

\bibitem{HiggsPostulate2} 
\href{https://journals.aps.org/prl/abstract/10.1103/PhysRevLett.13.508}{P. W. Higgs, 
"Broken symmetries and the masses of gauge bosons",
Phys. Rev. Lett. 13, pp. 508-509, 1964}

\bibitem{HiggsDiscovery1} 
\href{https://arxiv.org/abs/1207.7235}{S. Chatrchyan et al.,
"Observation of a new boson at a mass of 125 GeV with the CMS experiment at the LHC",
Phys. Lett. B716, pp. 30-61, 2012}

\bibitem{HiggsDiscovery2} 
\href{https://arxiv.org/abs/1207.7214}{G. Aad et al.,
"Observation of a new particle in the search for the Standard Model Higgs boson with the ATLAS detector at the LHC", 
Phys. Lett. B716, pp. 1-29, 2012}

\bibitem{FirstEvidence}
\href{https://ui.adsabs.harvard.edu/abs/1980ApJ...238..471R/abstract}{V.C. Rubin, W.K. Ford and N. Thonnard,
"Rotational properties of 21 SC galaxies with a large range of luminosities and radii, from NGC 4605 (R=4kpc) to UGC 2885 (R=122kpc)",
Astrophysical Journal 238, pp. 471-487, 1980}

\bibitem{RotationCurves}
\href{https://academic.oup.com/mnras/article/249/3/523/1005565}{K.G. Begeman, A.H. Broeils and R.H. Sanders,
"Extended rotation curves of spiral galaxies - Dark haloes and modified dynamics",
Monthly Notices of the Royal Astronomical Society, vol. 249, issue 3, ISSN 0035-8711, 1991}

\bibitem{BulletCluster}
\href{https://arxiv.org/abs/1605.04307}{A. Robertson, R. Massey and V. Eke,
"What does the Bullet Cluster tell us about self-interacting dark matter?",
Monthly Notices of the Royal Astronomical Society, vol. 465, issue 1, 2017}

\bibitem{CMBAnisotropies} 
\href{https://arxiv.org/abs/1804.01092}{J.B. Mu�oz, C. Dvorkin and A. Loeb,
"21-cm Fluctuations from Charged Dark Matter",
Phys. Rev. Lett. 121, 121301 (2018)}

\bibitem{Repartition}
\href{https://arxiv.org/abs/1201.3939}{A. Natarajan,
"A closer look at CMB constraints on WIMP dark matter",
Phys. Rev. D85, 2012}

\bibitem{MFVYukawa}
\href{https://arxiv.org/abs/hep-ph/0207036}{G. D'Ambrosio G.F. Giudice, G. Isidori and A. Strumia,
"Minimal Flavour Violation: an effective field theory approach",
Nucl.Phys. 645, pp 155-187, 2002}

%\bibitem{YukawaMeasurement}
%\href{https://arxiv.org/abs/1907.01590}{CMS Collaboration,
%"Measurement of the top quark Yukawa coupling from $t \bar t$ kinematic distributions in the lepton+jets final state in proton-proton collisions at $\sqrt{s}= 13$ TeV",
%Phys. Rev. D 100, 072007 (2019)}

\bibitem{PDG}
\href{http://pdg.lbl.gov/}{M. Tanabashi et al.,
Particle Data Group,
Phys. Rev. D98, 030001 (2018)}

%\bibitem{PreviousSingleTopAllLep2CDF}
%\href{https://arxiv.org/abs/1202.5653}{CDF Collaboration,
%"Search for a dark matter candidate produced in association with a single top quark in pp collisions at $\sqrt{s} = 1.96$ TeV",
%Phys. Rev. Lett. 108 (2012) 201802} MONOTOP!

\bibitem{PreviousDoubleTopSingleLep8CMS}
\href{https://arxiv.org/abs/1504.03198}{CMS Collaboration,
"Search for the production of dark matter in association with top-quark pairs in the single-lepton final state in proton-proton collisions at $\sqrt{s} = 8$ TeV",
JHEP, vol. 6 121, 2015}

\bibitem{PreviousDoubleTopDiLep8CMS}
\href{http://inspirehep.net/record/1292446}{CMS Collaboration,,
"Search for the Production of Dark Matter in Association with Top Quark Pairs in the Di-lepton Final State in pp collisions at $\sqrt{s} = 8$ TeV",
CMS-PAS-B2G-13-004, 2014}

\bibitem{PreviousDoubleTopAllLep8ATLAS}
\href{https://arxiv.org/abs/1410.4031}{
"Search for dark matter in events with heavy quarks and missing transverse momentum in pp collisions with the ATLAS detector",
Eur. Phys. J. C (2015) 75:92}

%\bibitem{PreviousSingleTopAllLep8ATLAS}
%\href{https://arxiv.org/abs/1410.5404}{ATLAS Collaboration,
%"Search for invisible particles produced in association with single-top-quarks in proton-proton collisions at $\sqrt{s} = 8$ TeV with the ATLAS detector",
%Eur. Phys. J. C (2015) 75:79} MONOTOP!

\bibitem{PreviousDoubleTopNoLep13ATLAS}
\href{http://inspirehep.net/record/1480057}{ATLAS Collaboration,
Search for the Supersymmetric Partner of the Top Quark in the Jets+Emiss Final State at $\sqrt{s} = 13$ TeV",
ATLAS-CONF-2016-077
}

\bibitem{PreviousDoubleTopOneLep13ATLAS}
\href{http://inspirehep.net/record/1480030/}{ATLAS Collaboration,
"Search for top squarks in final states with one isolated lepton, jets, and missing transverse momentum in $\sqrt{s} = 13$ TeV pp collisions with the ATLAS detector",
ATLAS-CONF-2016-050, 2016}

\bibitem{PreviousDoubleTopDiLep13ATLAS}
\href{http://inspirehep.net/record/1480056}{ATLAS Collaboration,
"Search for direct top squark pair production and dark matter production in final states with two leptons in $\sqrt{s} = 13$ TeV pp collisions using 13.3 fb$^{-1}$ of ATLAS data",
ATLAS-CONF-2016-076, 2016}

\bibitem{PreviousDoubleTopBottomAllLep13ATLAS}
\href{https://arxiv.org/abs/1710.11412}{ATLAS Collaboration,
"Search for dark matter produced in association with bottom or top quarks in $\sqrt{s} = 13$ TeV pp collisions with the ATLAS detector",
Eur. Phys. J. C 78 (2018) 18}

\bibitem{PreviousDoubleTopBottomAllLep13CMS}
\href{http://inspirehep.net/record/1603635}{CMS Collaboration,
Search for dark matter produced in association with heavy-flavor quark pairs in proton-proton collisions at $\sqrt{s} = 13$ TeV",
Eur. Phys. J. C (2017) 77: 845}

\bibitem{PreviousDoubleTopAllLep13CMS}
\href{https://arxiv.org/abs/1807.06522}{CMS Collaboration,
"Search for dark matter particles produced in association with a top quark pair at $\sqrt{s} = 13$ TeV",
Phys. Rev. Lett. 122, 011803 (2019)}

\bibitem{PreviousSingleDoubleTopAllLep13CMS}
\href{https://arxiv.org/abs/1901.01553}{CMS Collaboration,
"Search for dark matter produced in association with a single top quark or a top quark pair in proton-proton collisions at $\sqrt{s} = 13$ TeV",
JHEP, vol. 03 141, 2019}

\end{thebibliography}

\end{document}
